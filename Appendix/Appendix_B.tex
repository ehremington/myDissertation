

Application of quality electrical contacts is a critical element in the performance of electrochemical impedance spectroscopy (EIS) measurements. In the course of this work we assembled a simple magnetron sputtering system and optimized it for nickel (Ni) contact deposition, which is suitable for studying barium zirconate samples at temperatures below 500\textdegree C. This appendix is provided as a resource for future researchers of our facilities who may find the assembled system useful for their investigations.

Contacts need to adhere well throughout the contact area and also resist delamination from thermal stress during high temperature measurements. Ultimately contacts should be porous to allow gas phase element to be in close proximity to both the protonic conductor and electrical conductor. And the application of the contact must not compromise the integrity of the underlying oxide thin film. One of the most common and cost-effective techniques to apply porous contacts to oxide thin films is magnetron assisted D.C. sputtering. D.C. sputtering is a deposition technique in which a moderate potential difference ($\sim500$ V) is applied between the substrate and chamber walls (anode) and the metal target area (cathode). At argon pressures in the 10-50 mtorr range, a plasma is formed as the gas is ionized, and the positively charged Ar ions collide with the cathode plate. This ballistic collision sputters metal atoms and ions away and some of these atoms deposit on the substrate. 

Magnetron assisted sputtering achieves higher deposition rates at lower Ar pressure and potential difference. This is achieved by placing several magnets just under the metallic cathode in such a way that a magnetic field is oriented parallel to the surface of the cathode. Electrons emitted from the cathode or stripped from Ar atoms near the cathode experience a deflection due to the Lorentz force, increasing collisions with Ar atoms, and making a more dense plasma near the metal surface. 

In this manner, platinum (Pt) can be deposited over a thin film of BaZrO$_3$ with low enough energy to preserve the integrity of the film. Argon background pressure can be varied but pressures of 20 mtorr have been reported as optimal. A potential difference of 500 V has been used to good effect as well with other metals such as Ni, Cu, and Al, which may also be useful in our studies in conductivity measurements at lower temperatures.
