\vspace{12pt}
\section{Dopants and Defect Formation}Incorporating mobile protons into the perovskite structure is an obviously necessary step toward having proton conduction. The ideal dopant is a trivalent cation in the +4 valency B-site of the ABO$_3$ structure, which allows for proton incorporation by way of filling oxygen vacancies formed from dopant incorporation with hydroxyl groups. ``Doping'' in this context refers to the replacement of substantial fractions (typically 1-20\%) of the original B-site cation with the trivalent dopant ion. It is usual to represent doped barium zirconate by the formula BaZr$_{1-x}$M$_x$O$_{3-\delta}$ where M is a trivalent transition or rare-earth cation such as Y$^{3+}$, In$^{3+}$, Nd$^{3+}$, Gd$^{3+}$, or Yb$^{3+}$, and $\delta$ represents the fraction of oxygen vacancies formed. In the above chemical formula $\delta$ is related to the doping fraction $x$ by $x/2$ so that for every two dopant ions incorporated, one oxygen vacancy is formed. For the sake of simplifying the discussion, the following will proceed with Gd$^{3+}$ as the assumed dopant, but the others can be applied in a similar way. These oxygen vacancies form as the result of a charge compensation mechanism since the relative change in positive charges due to the substitution of +4 cations (Zr) with +3 cations (Gd) drives a corresponding amount of oxygen vacancies to maintain overall charge balance. So, since the dopant ions are less positive than the ion they replace, fewer negative charges are necessary to have an overall neutral charge. In the case of BaZrO$_3$ doped with gadolinium by incorporation of Gd$_2$O$_3$, Kr\"oger-Vink notation \cite{Kroger1956} describes this process as:
\begin{equation}
\mathrm{Gd_2 O_3 + 2 Zr_{Zr}^x + O_O^x \rightarrow 2 Gd_{Zr}' + V_O^{\bullet\bullet} + 2 Zr O_2}.
\end{equation}
As the above formula shows, gadolinium ions occupy zirconium positions, and its lower oxidation state is an effective negative charge relative to the zirconium that was replaced. This decrease in positive charge is compensated by the accompanying oxygen vacancy, such that the inclusion of two trivalent dopants are balanced by a single oxygen vacancy \cite{Islam2000}. 

However, an alternative ``reaction'' is also possible \cite{Ding2018, Han2014}. Critically, the stoichiometry and dopant size play a role in the formation of defects. Consider the following formula:
\begin{equation}
\mathrm{Gd_2 O_3 + 2 Ba_{Ba}^x + V_O^{\bullet\bullet} \rightarrow 2 Gd_{Ba}^{\bullet} + O_O^x + 2 Ba O}.
\label{eq:back:bariumSubstitution}
\end{equation}
In this case, the dopant cation substituting for barium will increase the overall charge and therefore be compensated by filling an oxygen vacancy in the process (one perhaps created by a Zr substitution nearby), thus working against the goal of proton incorporation and conduction. This effect can be the result of non-stoichiometry at the time of reaction (barium deficiency) or from high sintering temperatures causing loss of barium since the high vapor pressure of BaO at high temperatures causes evaporation \cite{Babilo2005}. This is a common problem in the synthesis of barium zirconate due to its highly refractory nature and the long sintering times necessary for bulk material to reach full density \cite{Haile2010, Tong2010}.

Once oxygen vacancies form from charge compensation, these vacancies can be filled provided they also maintain overall charge balance. Protons can thus be incorporated by exposure to water vapor, which dissociates into a hydroxide anion and a proton, replacing the oxygen vacancies with the hydroxide and incorporating the additional hydrogen by forming another hydroxyl group. This maintains overall charge balance in the lattice. In Kr\"oger-Vink notation this is written as: 
\begin{equation}
\mathrm{H_2 O (gas) + V_O^{\bullet\bullet} + O_O^x \leftrightarrow 2 OH_O^{\bullet}}
\end{equation}
These incorporated protons are free to migrate among oxygen sites. Therefore every two dopant ions incorporated in zirconium site produces one oxygen vacancy defect which then can facilitate two proton defects. Thus, the maximum hydration of the sample is the same as the doping concentration, which is commonly seen up to 20 mol\%. In barium cerate, increasing the dopant percentage in the case of yttrium causes structural changes that distort the lattice from cubic to rhombohedral \cite{Takeuchi2000} while also reaching a solubility limit at 17\% \cite{Giannici2007}. Increasing dopant levels up to 50\% have been shown to increase proton concentration in barium zirconate doped with yttrium, but this had a detrimental effect on conductivity due to structural distortions and proton trapping around dopant ions. Higher dopant concentrations also lead to increased substitution on Ba sites \cite{Fabbri2010c, Park2010}. So in the interest of maximizing conductivity, dopant levels of not more than 20 mol\% are used throughout the literature, while higher levels are used to study structural effects \cite{Han2016}.  