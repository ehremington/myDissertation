\vspace{12pt}
\section{Grain vs. Grain Boundary Conductivity}The refractory nature of barium zirconate, that is the high temperatures required to sinter, is another important factor in the measurement of conductivity that runs in parallel to the discussion of dopants. These high sintering temperatures make it difficult to grow large grains of bulk material without compromising the important stoichiometric balance. Early studies of barium zirconate dismissed it as a candidate for electrolyte materials due to the low conductivity ($\sim$10$^{-4}$ S/cm at 600\textdegree C) that was measured for it, especially in comparison to the cerates that were under heavy study at the time \cite{Iwahara1993, MANTHIRAM1993, Slade1995}. Missed in these studies were the difference between conductivity between the grains (inter-grain conductivity) and within the bulk of a grain (intra- or bulk conductivity) as well as the difficulty in consistent sample preparation and measurement technique. More careful study by Kreuer showed that the bulk conductivity was quite high while the conductivity across grain boundaries was very low \cite{Kreuer1999}. This fact coupled with the inability to easily grow large grained material explained the early results as well as indicated several research approaches to barium zirconate, and consequently interest in the material flourished. One focus has been to find suitable sintering agents, such as zinc \cite{Babilo2007} or nickel \cite{Tong2010} which would lower the temperatures necessary to sinter large grains of barium zirconate from 1600\textdegree-1800\textdegree C to 1300\textdegree C. As mentioned earlier, this would not only improve grain growth but also prevent barium loss and the consequent driving of dopant ions into its position. A second approach has been to find a marriage between barium zirconate and barium cerate or some other ion conductor, to combine the higher conductivity of the cerate while mitigating its lower chemical stability in CO\textsubscript{2} environments. A third approach, and the one of this dissertation, is to change the nature of the grain growth method entirely, from that of sintering a collection of grains, to a thin film deposition. In this method it is possible to grow the material layer by layer while still carefully controlling the stoichiometric requirements. As noted in the previous section, a number of research groups \cite{Pergolesi2010, Kim2011, Shim2008} all show outstanding conductivity results for yttrium doped barium zirconate deposited with pulsed laser deposition and large grain growth. Conductivity increases with grain size, and very high conductivity was reported for epitaxially grown grain boundary free thin films. 

Large grain growth using pulsed laser deposition could allow for the more careful investigation of the role of these dopants in this structure. No previous study has shown the deposition of gadolinium or ytterbium doped thin films, and no previous study has compared the conductivity of these dopants with those of yttrium.