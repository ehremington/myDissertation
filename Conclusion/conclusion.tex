This dissertation explored the synthesis of bulk and thin film configurations of doped barium zirconate. Bulk samples were prepared by solid state reaction with various dopants (Gd, Yb, Y). In the case of Gd, comparisons were drawn with the bulk material also produced by combustion spray pyrolysis. Dopant incorporation in the intended B-site of the perovskite structure was confirmed by two factors. First, the presence of intermediate compounds formed during solid state reaction that require the dopant to coordinate with oxygen atoms. Second, lattice constant changes were noted in samples that were doped with atoms of greater atomic radii, which is consistent with B-site incorporation. Overall, doping with species of greater ionic radii increased the lattice parameter of a BaZr$_{1-x}$M$_x$O$_3$ perovskite structure in the bulk samples. A similar correlation in the case of thin films deposited by PLD could not be ascertained, most likely due to stress in the thin films, which may have contributed to peak shifts that in the XRD patterns precluded the detection of a clear trend in lattice constant variations. Such determinations would have been straightforward in relaxed films.

An electrochemical impedance spectroscopy system was designed and implemented for use in high temperature environments with controls for gas type and humidity levels. Bulk samples produced in this research were characterized in terms of how their conductivity related to both the surrounding chemical environment and the incorporation of dopants into the structure as examined using X-ray diffraction (XRD). Changes in ambient gas during conductivity measurements showed clear differences in the conductivity of bulk samples. Evidence of protonic conductivity was documented for bulk samples measured in hydroxyl-rich environments. Dry environments (dry inert gas or vacuum) led to conductivities consistent with oxide-ion conduction. For exposure to ambient air, mixed electronic/ionic conductivity is likely to dominate. In the case of bulk samples, a correlation was measured between increasing ionic radius and overall conductivity. Y-, and Gd-doped bulk samples (which possess greater ionic radii), exhibited higher ionic conductivity than the Yb-doped material after heat treatment at 1600\textdegree C.

The increased ionic conductivity in both Y- and Gd-doped bulk samples may also be related to the more effective dopant incorporation than in the case of Yb. Such improved incorporation is also expected to yield an increased lattice parameter, as observed.

Thin films of doped barium zirconate were produced by pulsed laser deposition with Yb, Y, and Gd doping and characterized according to structure and conductivity. To the best of our knowledge, this is the first time such a study has been conducted on the deposition of thin films doped with gadolinium and ytterbium, as well as the first report of conductivity values for these materials in thin film form. Parameters for thin film growth by pulsed laser deposition were examined and the substrate temperature and laser energy density in particular stood out as important for highly textured growth. Dopant incorporation in the films was confirmed by chemical analysis. Comparative conductivity measurements under hydroxyl-rich environments vs. oxidizing and inert ambient environments, yielded robust evidence of protonic conductivity in the films. Thin films containing Gd exhibited the highest ionic conductivity values in comparison with the other dopants, which supports the original motivation of this dissertation research. From the analysis of both targets and thin films, gadolinium performed well as a dopant and warrants renewed interest. While in bulk form its conductivity was generally lower in comparison to previous studies \cite{Gilardi2017}, this may be due to it not having yet been optimized for bulk conductivity testing. The conductivity of both gadolinium and ytterbium barium zirconate thin films has not yet been reported in the literature, and offers good opportunities for peer-reviewed publications. Although their conductivities are lower than some reported studies of yttrium as a dopant, they are generally consistent with reported values. Refinement of our measurement methods and further testing may reveal gadolinium as a top choice of dopant.

The measured conductivities of bulk and thin films samples were consistent with similar
protonic materials in their temperature dependence and activation energies. Thin films with crystallographic texture exhibited substantially higher ionic conductivity than their randomly-oriented counterparts. Epitaxially-oriented films showed the highest conductivity values.

Numerous opportunities exist for furthering the research directions explored in this dissertation. Although effective in the epitaxially-driven growth of highly-oriented Gd-doped BaZrO$_3$ thin films, the MgO substrate is not suitable for electrochemical device applications. However, the PLD approach enables growth of textured thin films on other substrates without the demand of epitaxy. A systematic study of the various degrees of crystallographic texture possible on more suitable conductive and porous substrates, such as metal ceramic composites (e.g., BaZrO$_3$/Ni ``cermets’’) could yield ionic conductivities on par with the ones observed for the epitaxially-oriented Gd-doped material.

The enhanced ionic conductivity observed in the Gd-doped bulk and thin films (in comparison with their undoped as well as Y- and Yb-doped counterparts) suggests that further studies are warranted. Other investigations have shown a diversity of outcomes in this regard, some aligned with our inference while others opposing. Additional studies in the Gd-doped samples with further optimization in deposition and measurement may be able to determine the exact reason for the conductivity enhancement suggested by our samples.

In addition to further testing of other dopant or of co-dopant configurations, future studies should focus on several key areas. (1) Further development and testing should be done to make targets that are stoichiometrically balanced in terms of barium content or perhaps with an excess of barium content in order to perform well during the ablation process. In this dissertation, target production and optimization was a major bottleneck for making thin films that had reliable conductive properties. Optimization of target material and the processes to produce them could greatly expand the range of materials that can be tested as thin films. (2) Another major future study should be in alternative substrates that could allow measurement of conductivity of thin films in the ``through plane'' geometry. Whether these substrates be platinum with a titanium oxide interlayer to promote film growth, or on the aforementioned cermets that are purposefully composed to optimize barium zirconate growth, this is an area ripe for development. With this geometry and range of substrates, functioning fuel cells could then be tested. (3) Making functional fuel cells from these materials and expanding our group's capabilities to test fuel cell devices would be a natural extension of this project and would serve well the goal of combining materials of different properties (for example mixed conduction) to their highest use. This can be done by adapting our high temperature impedance probe to apply O$_2$ and H$_2$ in different chambers and measuring the resulting voltage and power output generated.  

Overall this dissertation explored both synthesis of new materials by previously unused methods as well as expanded the capabilities, knowledge base, and infrastructure to promote much more progress in the areas of ionically conducting materials and clean energy innovations.
