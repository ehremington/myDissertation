New developments and improvements across all sectors of energy generation, storage, transport, and use are necessary and good for society. The materials used to facilitate all of these actions are the subject of intense scrutiny as we expand our knowledge of their interactions and our control of their structure. Not only is exploration of new materials important but also the expansion of our capabilities to explore interactions and a wider variety of possibilities. 

The focus of this dissertation is on one particular material, barium zirconate, and its unusual properties to conduct ions, but it is also on the expansion of capabilities at UAB to explore even more through the techniques and apparatus that have been developed and implemented in this new and promising area of ion conducting solids. 

Barium zirconate may play a useful role in the improvement of a class of fuel cells known as solid oxide fuel cells. A fuel cell converts chemical energy stored in fuel directly to electricity, bypassing combustion as a step of traditional dynamo generators and alternators. This device has a distinct advantage over battery technology as the energy storage and replenishment comes from a fuel rather than requiring recharging. Studying this material and others in its class may produce usable fuel cells at a smaller scale and more widespread market penetration, which can have a dramatic effect on energy efficiency and undesirable emissions.

The approach of this dissertation to this field is both practical and allows for better scientific inquiry of these materials. Practically, fuel cells can be improved in two ways: (1) the discovery of new materials that promote higher conduction at lower temperatures than the current state of the art; and (2) making these materials thinner, even down to the nano-scale, can limit Ohmic losses and allow for small scale devices. From the scientific vantage point, examining these materials in the thin films configuration can provide an advantage of simplifying and controlling the structure of these materials to allow a more systematic approach to studying the transport mechanisms within them and the effect of dopants. Studying new materials as thin films achieves all of these goals. 

Both atomistic simulations and experimental results of protonic conductivity point directly at the need for an investigation of Gd- and Yb-doped barium zirconate. These dopants have different ionic radius, which significantly alters the interaction between the movable H$^{+}$ ion and the dopant cations. This study examines the synthesis of bulk and pulsed-laser-deposited doped barium zirconate to seek experimental clarification of the role of dopant cation radius on conductivity. Also studied are the incorporation of the trivalent dopant ion into the normally tetravalent ion site of zirconium (Zr$^{4+}$) and its impact on the protonic conductivity of the resulting doped perovskite. This characterization is achieved using primarily energy-dispersive x-ray spectroscopy to confirm dopant incorporation, x-ray diffraction to determine the crystallographic characteristics of the doped material, and electrochemical impedance spectroscopy to measure ionic and mixed conductivity and its temperature dependence.

Single crystal and epitaxial thin films are believed to be the ideal protonic conductor systems, although these configurations are rarely available for practical applications due to their stringent and high-cost fabrication requirements. Nevertheless, the role of microstructure in these films is an important factor in the conduction process, so evaluation of the deposition parameters that may prove favorable to the occurrence of crystallographic texture and epitaxial orientation is an important goal. Correlations of ionic conductivity as measured using impedance spectroscopy and microstructure as inferred from x-ray diffraction of thin films of Gd- and Yb-doped barium zirconate are discussed, with attention to the contribution of grain boundaries to the ionic transport observed.

This dissertation is organized under the following outline. 

The background and literature review in chapter \ref{ch:background} set up many of the important topics of the structure and ion conduction mechanism as well as fill in information about the choice of dopants and the effects this might have on conductivity. The history and prior work are important for understanding the motivation for exploring these particular dopants in this particular material above other candidates. 

Chapter \ref{ch:methods} details a variety of techniques used in the synthesis and characterisation of this material. Synthesis of thin films is carried out by pulsed laser deposition and characterization methods fall under the categories of structural and conductive. X-ray diffraction and electrochemical impedance spectroscopy, being the principal methods of these categories, are reviewed. 

Chapter \ref{ch:bulk} involves the synthesis and analysis of bulk samples prepared and characterized for use as source targets in the laser ablation process. Characterizing these materials is an important step as this serves as a point of comparison to the films produced from them. The effects of different types of dopant on the structure and analysis of those effects on the conductivity of these bulk samples is investigated.

Chapter \ref{ch:films} involves the synthesis and analysis of thin films prepared by pulsed laser deposition. Characterization of dopant incorporation, analysis of deposition rate, and crystallographic structure are compared over the deposition parameters of temperature and energy density. Finally the conductivity of these films and how it relates to structure and gaseous environment are examined.

This dissertation concludes with a summary of the research objectives accomplished and the major results from the analysis of these materials. A discussion of directions for future research projects is also included. 

Three appendices are included which contain detailed extensions to several points in the text. The first is a derivation of the brick layer model and a discussion of its implications to data analysis, relevant to the discussion of electrochemical impedance spectroscopy in Chapter \ref{ch:methods}. The second is an explanation of a magnetron-assisted DC sputtering device that was constructed for the deposition of metallic contacts as electrodes for conductivity measurements. The third details a chemical reaction method of synthesizing doped barium zirconate that was explored in the course of this project.