\vspace{12pt}
\section{Energy Dispersive X-Ray Spectroscopy}
We have also used a technique known as energy dispersive X-ray spectroscopy (EDX) to identify the chemical elements present in our samples. In this technique, a beam of electrons irradiates a sample, exciting electrons in the atoms of the material to higher energy shells. When the electrons return to their ground energy state, their respective atoms emit an X-ray with energy corresponding to the difference in the energy shells. This quantized energy reveals the identity of the material. The intensity of the signal at a particular energy is proportional to the amount of that substance present in the sample. We performed EDX on our samples inside a scanning electron microscope (SEM). The specific SEM used was the Quanta 650 SEM (FEI, USA) with a field emission gun. The accelerating voltage used was 15 kV. The use of this instrument also allowed us to take SEM images of our samples.

