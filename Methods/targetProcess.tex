\section{Methods for preparation of doped barium zirconate in bulk form}
The synthesis and processing of barium zirconate has long been recognized as a challenge because of the thermophysical properties of this compound. Its very high melting temperature (above 2500\textdegree C) and very low expected values of surface and grain boundary self-diffusion coefficients make the effective sintering of compacted barium zirconate powders difficult.

Powder metallurgy techniques have focused on improving the sinterability of high-temperature ceramics, such as barium zirconate, for decades. Although a full review of the powder processing of barium zirconate is beyond the scope of this dissertation, we highlight a few of the techniques explored in this research to produce bulk barium zirconate samples with the basic characteristics needed for assessing ionic conductivity. Our primary focus is on sintering-agent-free approaches, since the common route of adding foreign chemical species to a high-temperature precursor powder, is typically deleterious to ionic conductivity and significantly complicates the elucidation of physical mechanisms involved in ionic conduction processes. Moreover, the possibility of laser-induced transformations akin to sintering, particularly in the context of thin film fabrication, developed later in this dissertation, may open substantially new and improved ways of addressing the sintering problem in ceramics.

Briefly, essentially two variations of powder metallurgy approaches have been explored in this project:  (1) Solid state reaction of barium carbonate, zirconium oxide and dopant oxide, and (2) Chemical solution techniques based on carbohydrates and metal salts.

These two approaches feature significantly high temperatures and times to achieve acceptable densification of the powder mixtures, which is a shared disadvantage of our strategies. On the other hand, they are both essentially free of added sintering agents, which has the potential of reducing the complexity of electrical data analysis. Detailed descriptions of these bulk sample preparation methods are provided as part of Chapter \ref{ch:bulk}. 