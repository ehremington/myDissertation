\section{Introduction}
The synthesis of doped barium zirconate in the form of thin films offers numerous opportunities for research in the fundamentals of transport processes in this class of materials as well as in potential applications. In the case of SOFCs, for example, electrolytes that feature both high ionic conductivity and reduced thickness, are expected to deliver significant performance improvements in the intermediate temperature range (500-750\textdegree C). In addition to the potential benefits to large-scale SOFC stacks, the lowered operating temperature of thin-film electrolytes is spawning the new research sub-field of “micro-SOFCs.” In contrast to power outputs in the hundreds of kilowatts of the SOFC stacks described earlier, micro-SOFCs are miniaturized devices that can function as small-scale energy supplies with higher energy per volume and weight than batteries. Applications for these units, which could potentially allow instant refueling, eliminating long charging times associated with batteries, have been proposed to numerous small-scale systems ranging from portable electronics to auxiliary power units in automobiles.

The pulsed laser deposition (PLD) technique, which is described in detail in section \ref{section:PLD}, was used to fabricate doped barium zirconate thin films in this project. The PLD approach allows access to many parameters that are relevant in altering the properties of a thin film ionic conductor. Chief among these are the microcrystalline structure of the film and its thickness. The films grown by PLD are later characterized in this dissertation employing XRD, in order to discern the crystalline versus amorphous nature of the films, as well as to determine the lattice parameters of the crystalline ones. The microstructure of the films varies with the deposition temperature and comparing relative intensity of XRD peaks allows the detection of crystallographic texture. In addition to structure, the thickness of the films deposited by PLD was measured with spectrophotometry, by comparing the thin film interference of different wavelengths of light. Transmission characteristics under different incident angles also enabled measurement of the index of refraction for the doped material. 